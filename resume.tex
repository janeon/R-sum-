\documentclass{resume}
\usepackage[left=0.6in,top=0.6in,right=0.6in,bottom=0.6in]{geometry}
\usepackage{hyperref}
\newcommand{\tab}[1]{\hspace{.266\textwidth}\rlap{#1}}
\newcommand{\itab}[1]{\hspace{0em}\rlap{#1}}
\name{Jane Hsieh}
\address{469-450-7176 \\ jane.hsieh@oberlin.edu}
\address{\href{https://janeon.github.io}{janeon.github.io }\\
\href{https://www.linkedin.com/in/jane-hsieh/}{linkedin.com/in/jane-hsieh}}
% \\ instagram.com/janeon32}
\begin{document}

\begin{rSection}{Education}
{\bf PhD in Software Enginnering from Carnegie Mellon University} \hfill { August 2020 - Present} \\ 
{\bf Bachelor of Arts from Oberlin College} \hfill { August 2016 - May 2020} \\
Majors in Computer Science and Mathematics, Concentration in Cognitive Sci. \hfill {(Major) GPA: (3.77) 3.71}
% \\ Relevant coursework:
% {AI, HCI, Advanced Algorithms, Nonlinear optimization, Systems Programming, Linear Algebra, Data Structures, Mathematical Modeling, Theory of Computation, Comp. Architecture}
\end{rSection}
% \begin{rSection}{Courses}
%  {Artificial Intelligence, Theory of Computation, Computer Architecture, Economics} \hfill {\em Current} \\
%  {Economics and Computation, HCI, Foundations of Analysis, Linguistic Anthropology} \hfill {\em Fall 2018}
%  {Advanced Algorithms, Systems Programming, Group Theory, Mathematical Modeling} \hfill {\em Spring 2018}
%  Algorithms, Linear Algebra, Data Structures, Discrete Math, Multivariable Calculus \hfill {\em 2017} \\
%  {Intro to Computer Science, Deconstructing Technology, Calculus II}\hfill {\em Fall 2016}
% \end{rSection}
%    Projects And Seminars
%-----------------------------------------------------------------------------------------------
\begin{rSection}{Experiences}
{\bf Support Portal at IBM} \hfill  {\em Summer 2020} \\
{\it Software Enginner Intern on IBM's Toolbox Team} \hfill { Raleigh, NC} \\
{Conducted user research with administrators to reveal internal productivity painpoints and devise solutions} \\
{Developed drivers in Slack and Github to provide self-service features that normally require admin privileges}

\href{https://github.com/IBM/multicloud-incident-response-navigator}{\bf Interactive Terminal Application for IBM's Multicloud Manager} \hfill  {\em Summer 2019} \\
\href{https://www.ibm.com/employment/extremeblue/index.html}{\em Extreme Blue Technical Intern, managed by Ross Grady} \hfill { Raleigh, NC} \\
Conducted user research and developed vi-based tool for multicloud applications using \underline{Python's curses} library\\
{Co-created the \href{https://github.com/IBM/multicloud-incident-response-navigator}{multicloud-incident-response-navigator project}, open-sourced on \href{https://github.com/IBM/multicloud-incident-response-navigator}{IBM's public cloud}} 

{\bf \href{https://unakite.info/}{UNAKITE Chrome Extension}}\hfill {\em Summer 2018-2019} \\
{\em \href{https://www.cmu.edu/scs/isr/reuse/}{REUSE Program at Carnegie Mellon University}, advised by Brad Myers \& Aniket Kittur} \hfill { Pittsburgh, PA}\\
Conducted user studies at the HCI institute, designed and implemented interface improvements using \underline{React} \\
Published and presented findings at 2018 {\em VL/HCC} and continued research through remote collaboration

{\bf Characterizing and Separating Magnetic Nanoparticles } \hfill {\em 2016 - 2018}\\
{\em \href{https://www.oberlin.edu/undergraduate-research/programs/strong}{STRONG Pre-First-Year Program}, advised by Yumi Ijiri } \hfill { Oberlin, OH}\\
Assisted in making design improvements for a nanoparticle separation channel after testing with a prototype \\
Used \underline{Jupyter Notebook} to fit polarization-analyzed small-angle neutron-scattering data from 16 conditions \\
Analyzed and presented findings on behavior and interactions of manganese ferrite nanoparticles

 \textbf{Technical languages:} {Python, React \& AngularJS, LaTex, Git, Java, {$C^{++}$}, CSS/HTML, Bash, Shell, Swift}
\end{rSection}
% \begin{rSection}{Publications}
% {\href{http://www.cs.cmu.edu/~NatProg/papers/p67-liu-Unakite-UIST.pdf}{\bf UNAKITE: Scaffolding Developers’ Decision-Making Using the Web} ~ \underline{Best Paper Honorary Mention}} \\
% \href{http://www.cs.cmu.edu/~NatProg/papers/p305-hsieh.pdf}{\bf An Exploratory Study of Web Foraging to Understand and Support Programming Decisions }\\
% \normalsize
% {\href{https://janeon.github.io/assets/img/PhysRevB.99.094421.pdf}{\bf Correlated spin canting in ordered core-shell Fe$_3$O$_4$/Mn$_x$Fe$_{3-x}$O$_4$ nanoparticle ... assemblies} \\
% {\bf UNAKITE: Support Developers for Capturing and Persisting Design Rationales ...} }
% \end{rSection} \normalsize 

\begin{rSection}{Volunteering}
{\bf \href{https://www.oc2020.oberlincollegelibrary.org/}{Web Development for Digital Yearbook}} \hfill{\em Summer 2020} \\
Designed and implemented visual layout to the digital yearbook using Omeka Classic, CSS, HTML and PHP

\href{http://www.cs.oberlin.edu/~csmc/officers.php}{\textbf{Computer Science Majors Committee Member}} \hfill {\em Fall 2018 - current} \\
Organized department activities, updated committee websites, held weekly office hours

\end{rSection}

\begin{rSection}{Activities \& Awards}
{\bf IBM North America 2020 Intern Hackathon} \hfill {\em Summer 2020} 

{\bf Teaching Assistant (office hour holder, grader, lab helper, dedicated tutor)} \hfill {\em Spring 2017 - current} 

% {\bf Grader for Data structures and algorithms}\hfill{\em 2017 - 2019} \\
% Assessed and provided feedback to student worksheets and programs.
% {\bf Lab helper for Introduction CS}\hfill {\em Spring 2017, 2019} \\
% Assisted a class of around 20 students in debugging and finding logical errors in weekly Python projects
% in \underline{Python} and \underline{Java}.
% \textbf{Teaching Assistant for Advanced Chinese}
% \hfill {\em Spring 2017} \\
% Led weekly oral practice sessions for a class of 8 students.
% {\bf Lap helper for Introduction to Computer Science} \hfill {\em Fall 2017 and Spring 2019} \\
% - Assisted ~20 students in weekly Python projects by helping debug and finding logical errors

% \textbf{Developed food optimization and messaging apps in} \underline{Swift}\hfill {\em PennApps 2018 \& Winter 2017}

% \textbf{Natural languages}: Mandarin, Shanghainese, Spanish \\
% {Other interests: Violin, rock climbing, reading}

{\bf \href{https://www.oberlin.edu/admissions-and-aid/for-accepted-students/virtual-visits/covid-19-course}{\textit{Uncovering Covid} Course Workshop Leader}} \hfill{\em Spring 2020} 

{\bf 2020 Annual R.J. Thomas Award for an Outstanding Computer Science Student} \hfill{\$500} 

{\bf \href{https://www.oberlin.edu/career/set/soar/soar-leaders}{SOAR (Sophomore Opportunities \& Academic Resources) Leader}} \hfill{\em Fall 2019 - Spring 2020} 



{\bf Clare Boothe Luce Scholarship at Oberlin College} \hfill{\em Fall 2018 - Spring 2019}

% Awarded per annum to a woman studying in a scientific field who intends to pursue graduate studies

{\bf Honorary Mention 2017 in ACM ICPC East Central NA Regional Contest}\hfill{\em Fall 2017}

% {\bf John F. Oberlin Scholarship}\hfill{\em 2016 - current}

% {\bf STRONG Scholarship \& IB Diploma recipient }\hfill{\em Summer 2016-2017} \\
%  Researched in 2016 and mentored 2017's cohort of students from underrepresented backgrounds
{\bf PennApps XVIII Hackathon}\hfill{\em Fall 2018}
\end{rSection}

\end{document}
