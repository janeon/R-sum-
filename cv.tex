\documentclass{resume}
\usepackage[left=0.6in,top=0.6in,right=0.6in,bottom=0.6in]{geometry} 
\usepackage{hyperref}

\newcommand{\tab}[1]{\hspace{.266\textwidth}\rlap{#1}}
\newcommand{\itab}[1]{\hspace{0em}\rlap{#1}} 
\name{Jane Hsieh} 
\address{469-450-7176 \\ Email: jhsieh@oberlin.edu}
\address{\href{https://janeon.github.io}{janeon.github.io }\\
\href{https://www.linkedin.com/in/jane-hsieh/}{linkedin.com/in/jane-hsieh}
% \\ instagram.com/janeon32
} 
\begin{document}
\begin{rSection}{Education}
{\bf PhD in Software Enginnering from Carnegie Mellon University} \hfill { August 2020 - Present} \\ 
{\bf Bachelor of Arts from Oberlin College} \hfill { August 2016 - May 2020} \\ 
Majors in Computer Science and Mathematics, Concentration in Cognitive Science \hfill {Major GPA: 3.77} \\
Member of {\em IEEE} and {\em American Physical Society} \hfill {Cumulative GPA: 3.71} \\
IB Diploma recipient \hfill {HS Ranking: 2/22}
\end{rSection}
% \begin{rSection}{Courses}
%  {Artificial Intelligence, Theory of Computation, Computer Architecture, Economics} \hfill {\em Current} \\
%  {Economics and Computation, HCI, Foundations of Analysis, Linguistic Anthropology} \hfill {\em Fall 2018} 
%  {Advanced Algorithms, Systems Programming, Group Theory, Mathematical Modeling} \hfill {\em Spring 2018} 
%  Algorithms, Linear Algebra, Data Structures, Discrete Math, Multivariable Calculus \hfill {\em 2017} \\
%  {Intro to Computer Science, Deconstructing Technology, Calculus II}\hfill {\em Fall 2016}
% \end{rSection}

%    Projects And Seminars
%-----------------------------------------------------------------------------------------------

\begin{rSection}{Research \& Work, Experiences}
{\bf Support Portal at IBM} \hfill  {\em Summer 2020} \\
{\it Software Enginner Intern on IBM's Toolbox Team} \hfill { Raleigh, NC} \\
{Conducted user research with administrators to reveal internal productivity painpoints and devise solutions} \\
{Developed drivers in Slack and Github to provide self-service features that normally require admin privileges}

\href{https://github.com/IBM/multicloud-incident-response-navigator}{\bf Interactive Terminal Application for IBM's Multicloud Manager} \hfill  {\em Summer 2019} \\
\href{https://www.ibm.com/employment/extremeblue/index.html}{\em Extreme Blue Technical Intern, managed by Ross Grady} \hfill { Raleigh, NC} \\
{Conducted user research with internal Kubernetes operators to identify relevant painpoints} \\
Developed vi-based tool for multicloud applications using Python's curses library and Agile practices \\
{Created the open-sourced \href{https://github.com/IBM/multicloud-incident-response-navigator}{multicloud-incident-response-navigator project}} and \href{https://priorart.ip.com/IPCOM/000262660}{published patent defense}

{\bf \href{https://unakite.info/}{UNAKITE Chrome Extension}}\hfill {\em Summer 2018-2019} \\
{\em \href{https://www.cmu.edu/scs/isr/reuse/}{REUSE Program at Carnegie Mellon University}, advised by Brad Myers \& Aniket Kittur} \hfill { Pittsburgh, PA}\\
% Using Agile and design thinking practices, I developed an interactive terminal tool in with Python's curses library to help users manage Kubernetes applications. In particular, I worked with IBM's Multi-cloud Manager to create the multicloud-incident-response-navigator project that is now open-sourced in IBM's public cloud.
Conducted user studies at the HCI institute, designed and implemented interface improvements using React \\
Published and presented findings at the 2018 {\em VL/HCC} conference \\
Continued various user studies and analysis through remote collaboration

{\bf Characterizing and Separating Magnetic Nanoparticles } \hfill {\em 2016 - 2018}\\
{\em \href{https://www.oberlin.edu/undergraduate-research/programs/strong}{STRONG Pre-First-Year Program}, advised by Yumi Ijiri } \hfill { Oberlin, OH}\\
Assisted in making design improvements for a nanoparticle separation channel after testing with a prototype \\
Used Jupyter Notebook to fit polarization-analyzed small-angle neutron-scattering data from 16 conditions \\
Analyzed resulting trends to learn about behavior and interactions of the manganese ferrite particles 

\end{rSection}

\begin{rSection}{Publications}
\begin{enumerate}
    \item Michael Xieyang Liu, \textbf{Jane Hsieh}, Nathan Hahn, Angelina Zhou, Emily Deng, Shaun Burley, Cynthia Taylor, Aniket Kittur, Brad A. Myers, ``Unakite: Scaffolding Developers’ Decision Making About Trade-offs through Capturing and Organizing Web Resources", \textit{ACM Symposium on User Interface Software and Technology, UIST'19}, New Orleans, LA, October 20-23, 2019. pp. 67-80. \href{https://dl.acm.org/citation.cfm?id=3347908}{ACM DL} and \href{http://www.cs.cmu.edu/~NatProg/papers/p67-liu-Unakite-UIST.pdf}{local pdf}.\\
    \textbf{Best Paper Honorable Mention Award} from the ACM Symposium on User Interface Software and Technology, UIST'19 (top 6 out of 93 accepted papers). 
    \item Michael Xieyang Liu, Nathan Hahn, Angelina Zhou, Shaun Burley, Emily Deng, \textbf{Jane Hsieh}, Aniket Kittur and Brad A. Myers, ``UNAKITE: Support Developers for Capturing and Persisting Design Rationales When Solving Problems Using Web Resources", \textit{DTSHPS'18 Workshop on Designing Technologies to Support Human Problem Solving} (\href{https://www.cs.washington.edu/dtshps2018/index.html}{DTSHPS'18}) at VL/HCC'2018. Oct. 1, 2018. p. 25. \href{http://www.cs.cmu.edu/~NatProg/papers/DTSHPS%20paper%207%20-%20one-page-summary-with-references%20v2.pdf}{extended abstract} or \href{https://digital.lib.washington.edu/researchworks/bitstream/handle/1773/42857/DTSHPS18-Proceedings-final%20v2.pdf}{full proceedings}.
    \item \textbf{Jane Hsieh}, Michael Xieyang Liu, Brad A. Myers, Aniket Kittur, ``Poster: An Exploratory Study of Web Foraging to Understand and Support Programming Decisions," \textit{2018 IEEE Symposium on Visual Languages and Human-Centric Computing} (VL/HCC'18), October 1 - 4, 2018, Lisbon, Portugal. pp. 305-306. \href{https://ieeexplore.ieee.org/document/8506517}{IEEE DL} and \href{http://www.cs.cmu.edu/~NatProg/papers/p305-hsieh.pdf}{local pdf}.
    \item Yumi Ijiri, Kathryn L. Krycka, Ian Hunt-Isaak, Hillary Pan, \textbf{Jane Hsieh}, Julie A. Borchers, James J. Rhyne, Samuel D. Oberdick, Ahmed Abdelgawad, Sarah A. Majetich, ``Correlated spin canting in ordered core-shell Fe$_3$O$_4$/Mn$_x$Fe$_{3-x}$O$_4$ nanoparticle polycrystalline assemblies," \textit{Physical Review B} 99(9). March 18, 2019. p. 094421. \href{https://journals.aps.org/prb/abstract/10.1103/PhysRevB.99.094421}{APS DL} and \href{https://janeon.github.io/assets/img/PhysRevB.99.094421.pdf}{local pdf}.

\end{enumerate}

% \fontsize{10.25}{12}\selectfont{\bf Correlated spin canting in ordered core-shell Fe$_3$O$_4$/Mn$_x$Fe$_{3-x}$O$_4$ nanoparticle polycrystalline assemblies}
% \fontsize{11}{12}\selectfont
% { Full paper in Physical Review B} \hfill {\em American Physical Society 2018}
% {\bf Scaffolding Developers’ Decision-Making Using the Web} \\ { Full-paper, second author, \underline{honorary mention} \hfill  {\em ACM UIST Symposium 2019}}



% {\bf An Exploratory Study of Web Foraging to Understand and Support Programming Decisions} \\
% { First author, poster \& extended abstract} \hfill {\em Visual Learning and Human-Centric Computing 2018}


\end{rSection}
\begin{rSection} {Conferences \& Workshops Attended}

{\bf Symposium on Visual Language and Human-Centric Computing} \hfill {October 2018} \\
Presented poster and short talk. Submitted extended abstract and workshop paper \hfill {\em Lisbon, Portugal}

{\bf Grace Hopper Celebration} \hfill {Fall 2018} \\
Research scholar with the Computing Research Association for Women \hfill {\em Houston, Texas}

{\bf Ohio Summer Research Symposium} \hfill {July 2017}\\
Gave talk on modeling PASANS data of Manganese Ferrite Nanoparticles \hfill {\em Ohio Wesleyan University}

{\bf Celebration of Undergraduate Research} \hfill {Oberlin, OH} \\
 {- Poster: An Exploratory Study of Web Foraging to Understand \& Support Programming Decisions}\hfill {\em 2018} 
 
 {- Poster: Determining the Magnetic Structure of Ferrite Nanoparticles} \hfill {\em 2017} 
 
 {- Poster: Improving the Design of a Magnetic Nanoparticle Separation Channel} \hfill {\em 2016}
\end{rSection}
\begin{rSection}{Awards \& Honors} 
{\bf 2020 R.J. Thomas Award for Outstanding Computer Science Student} \hfill{\$500} \\
Awarded per annum to one senior in the Computer Science Department

{\bf Clare Boothe Luce Scholarship at Oberlin College} \hfill{\$38,808} \\
Awarded per annum to a woman studying in a scientific field

{\bf 2018 Computing Research Association for Women GHC Research Scholarship}\hfill{\$500} 

{\bf John F. Oberlin Scholarship } \hfill{\$69,000}

{\bf Oberlin College Grant } \hfill {\$23,538}

{\bf Oberlin ASG Endowed Scholarship} \hfill {\$20,324}

{\bf STRONG Scholar}\hfill{\$2,500} \\
 Researched in 2016 and mentored 2017's cohort of students from underrepresented backgrounds
\end{rSection}

\begin{rSection}{Teaching, Extracurriculars \& Volunteering}
{\bf \href{https://www.oc2020.oberlincollegelibrary.org/}{Web Development for Digital Yearbook}} \hfill{\em Summer 2020} \\
Designed and implemented visual layout to the digital yearbook using Omeka Classic, CSS, HTML and PHP

{\bf \href{https://www.oberlin.edu/admissions-and-aid/for-accepted-students/virtual-visits/covid-19-course}{Uncovering Covid Workshop Leader}} \hfill{\em Spring 2020} \\
Planned, trained for and led weekly discussions for 15 admitted Oberlin students on a half-module course exploring Covid-19 from a variety of disciplines. Attended weekly lectures by professors from 8 departments.

{\bf \href{https://www.oberlin.edu/career/set/soar/soar-leaders}{Sophomore Opportunities \& Academic Resources (SOAR) Leader}} \hfill{\em Fall 2019 - current} \\
Recruit participants and plan for winter retreat to provide students with resources for major declaration

{\bf Office hour holder and Tutor for Algorithms} \hfill {\em Fall 2018} \\
{ Led group workshops to guide students on homework problems twice per week (sessions open to entire class)} 

{\bf Grader for Algorithms, Data Structures} \hfill {\em Fall 2017 - Fall 2018} \\
{ Assessed and provided feedback to  $\approx$ 20 student worksheets weekly} 

\href{http://www.cs.oberlin.edu/~csmc/officers.php}{\textbf{Computer Science Majors Committee Member}} \hfill {\em Fall 2018 - current} \\
{ Organized department activities, updated committee websites, held weekly office hours} 

{\bf Lab helper for Introductory course in Python} \hfill {\em Spring 2017, 2018} \\
{ Assisted $\approx$ 20 students debug and find logical errors in weekly Python assignments}

{\bf Oberlin Workshop \& Learning Sessions (OWLS) Leader for Algorithms} \hfill {\em Fall 2018} \\
{ Attended class to plan and lead interactive, non-traditional workshops (weekly)} 

{\bf ACM ICPC East Central NA Regional Contest}\hfill{\em Fall 2017} \\
{Received Honorary Mention} 

{\bf Advanced Chinese Drill Session Teacher} \hfill {\em Spring 2017} \\
{ Created lesson plans (after attending class) to lead weekly drills to help students improve speaking fluency}

\textbf{Technical languages:} {Python, Javascript (React \& Angular), LaTex, Git, Java, {$C^{++}$}, CSS/HTML, Swift}


\textbf{Spoken languages: }{Mandarin, Shanghainese, Spanish} 

{\bf Other interests: }{Violin, running, rock climbing, baking, reading}
\end{rSection}

\begin{rSection} {Other Projects}
{\bf Automated Lab Helper
% (AI final project)
}\hfill  {\em Spring 2019} \\
Created program that lints code, sorts errors and recommends solutions for beginning CS students at Oberlin

{\bf Frontend Dev for Conceptum: a Question Repository for Educators}\hfill  {\em Winter - Spring 2019} \\
Implemented Angular interface components for an iterative question development site designed for professors

{\bf Taskat 
% (HCI final project)
}\hfill  {\em Fall 2019} \\
Designed and implemented {React} Electron desktop app to help users to record, and track time of tasks

{\bf Star and Galaxy Clustering 
% (Systems Programming final project)
}\hfill {\em Spring 2018} \\
{Implemented K-means in {$C^{++}$}, used SIMBAD catalogue to query $\sim$ 1000 stars and gnuplot as frontend}

\textbf{Food Optimization and Peer Tutoring Messaging apps}\hfill {\em Fall 2018 \& Winter 2017} \\
{ Developed prototype iOS apps using Swift 2 \& 3} \hfill { PennApps \& Oberlin}
\end{rSection}

\end{document}
